\documentclass[a4paper]{article}

\usepackage[english]{babel}
\usepackage[utf8]{inputenc}
\usepackage[T1]{fontenc}
\usepackage{sectsty}
\usepackage{fullpage}
\usepackage{syntax}
\usepackage{array}
\usepackage{multirow}
\usepackage{amsmath}
\usepackage{comment}
\usepackage{graphicx}
\usepackage{hyperref}

\hypersetup{
     colorlinks   = true
}

\sectionfont{\fontsize{12}{15}\selectfont}

\newcommand{\BigO}[1]{\ensuremath{\operatorname{O}\left(#1\right)}}

%%%%%%%%%%%%%%%%%%%%%%%%%%%%%%%%%%%%%%%%%%%%%%%%%%%

\title{\textsc{WebApps}\\Project Report}

\author{Tom Burnell \{thb12\} \  Andrea Michi \{am8112\} \\
        Lawrence Jones \{lmj112\}}

\date{}
\begin{document}
\maketitle

%%%%%%%%%%%%%%%%%%%%%%%%%%%%%%%%%%%%%%%%%%%%%%%%%%%%%%%%%%%%%%%%%%%%%%%%%%
%% QUESTION 1
%%%%%%%%%%%%%%%%%%%%%%%%%%%%%%%%%%%%%%%%%%%%%%%%%%%%%%%%%%%%%%%%%%%%%%%%%%

CATe is a Virtual Learning Environment system currently in place for DoC students at Imperial College. The site supports distribution of course content, assignment of exercises/courseworks and disseminate student grades.

While CATe is still performing admirably for the department's purposes, the site is now showing age with respect to dated UI/UX, and more importantly a lack of interaction with important student information.

In summer 2012, a DoC student Peter Hamilton created a web userscript \textit{(\href{https://github.com/petehamilton/classy-cate}{classy-cate})} that could be installed with Tampermonkey or equivalent, that reskinned CATe and dealt with scraping of information. This went on to be developed by DoC students, with a few features such as upcoming deadlines presented on a dashboard. Unfortunately not many students used the script, mainly due to lack of awareness of how userscripts work.

On top of this, due to the nature of a webscript, the projects aim was limited without full control over the clientside and no ability to collect/persist data. As such, the list of possible enhancements are restricted. The logical step was to move from userscript to a separate application entirely, in order to allow a wider scope of enhancements to be built on top of CATe.


%% APP DESCRIPTION
\section{App Description}

Our app is to be a website that makes use of CATe data to provide a richer VLE resource for DoC students. Our key aims are to reduce the current frustrations with CATe and address the issue of student interaction and discusion on courses.

CATe organises data in a slightly scattered manner; course notes and grades are detached from courses, and exam information with past papers is hosted on an entirely different domain.

We plan to make a DoC course an integral part of our site, hosting data around those courses in order to group relevant content for easier access.

The division of exams and course content prevents useful sharing of information, and so we aim to unify these two sources.

Finally, our site should promote student interaction with both the course and the lecturer who leads it. Exercises and lectures can benefit from student discussions, and while Piazza provides a platform the experience could be much improved by linking documents/content directly to the discussion thread. We aim to allow tagged discussion of our sites content, facilitating student collaboration over shared resources such as model answers.


%% PROJECT MANAGEMENT
\section{Project Management}

\subsection{Group Structure}
\subsection{Language/Framework Choices}
\subsection{Version Control \& Issue Tracking}

%% IMPLEMENTATION DETAILS
\section{Implementation Details}

\subsection{Server}

\subsection{Web Client}

\subsection{Jakefile}

\subsection{Parser-Proxy Pattern}

%% ACKNOWLEDGEMENTS
\section{Acknowledgements}

%% CONCLUSION
\section{Conclusion}

Achieved/Would like to/Would do differently

\end{document}
